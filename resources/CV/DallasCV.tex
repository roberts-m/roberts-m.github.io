%!TEX TS-program = xelatex
\documentclass[]{CV}
\setcounter{secnumdepth}{-1}
\usepackage{multicol}
\usepackage{fontawesome}
\usepackage{hyperref}
\usepackage{fontspec}


%\renewcommand{\familydefault}{\sfdefault}

\begin{document}

\header{}{Tad Dallas}{}


\begin{aside}
  \section{about}
    ~
		Assistant professor
		Louisiana State University
		Dept of Biological Sciences
    ~
    \faEnvelope \ \  \href{mailto:tad.a.dallas@gmail.com}{tad.a.dallas@gmail.com}
    \faDesktop  \ \ \href{https://taddallas.github.io}{taddallas.github.io}
    \faGithub   \ \ \ \href{http://github.com/taddallas}{taddallas}
    ~
  \section{programming}
   ~
   {\mefont Proficient}
   R
   Matlab/Octave
   SQL
   ~
   {\mefont Familiar }
   C++
   Julia
   Python
   ~
   {\mefont Markup}
   \LaTeX
   Markdown
   HTML/XML/XPath
   ~
   {\mefont Version control}
   git
\end{aside}





\section{\faFlask \ \ experience}

\begin{entrylist}

  \entry
    {2019 - }
    {Assistant professor}
    {Louisiana State University, \textit{Baton Rouge}}
    {\emph{Dept. of Biological Sciences}} \\


  \entry
    {2019}
    {Visiting researcher}
    {International University of Rijeka, \textit{Croatia}}
    {\emph{Dept. of Mathematics}} \\

  \entry
    {2019}
    {Visiting researcher}
    {CSIC, \textit{Estación Biológica de Do\~nana}, Spain}
    {\emph{Advised by Pedro Jordano}} \\

  \entry
    {2018 - 2019}
    {Postdoctoral fellow}
    {University of Helsinki - \textit{Centre for Ecological Change}}
    {\emph{Advised by Otso Ovaskainen}} \\


  \entry
    {2016 - 2018}
		{Postdoctoral fellow}
    {University of California--Davis - \textit{Center for Population Biology}}
    {\emph{Advised by Alan Hastings}} \\


  \entry
    {2015}
    {Distributed $R$ Analytics Intern}
    {HP Vertica - Big Data Platform Dev Team}
    {\emph{Software development for analysis of large data}} \\

  \entry
    {2010-2011 \ \ \ \ \ }
    {Biological Science Technician}
    {USDA - Agricultural Research Service}
    {\emph{Subtropical Plant Pathology Lab }} \\

  \entry
    {2008}
    {Mathematical Biology Program}
    {NSF Research Experience for Undergraduates (REU)}
    {\emph{Mathematical estimation of host range using mark-recapture data}}
\end{entrylist}





\section{\faGraduationCap \  education}

\begin{entrylist}
  \entry
    {2011 - 2016}
    {\normalfont Ph.D. Ecology}
    {U Georgia - Odum School of Ecology}
    {\emph{Advised by John Drake}}\\

  \entry
    {2009 - 2010}
    {M.S. Biology}
    {Truman State University}
    {\emph{Ecology of small mammal-tick interactions} \\ advised by Stephanie For\'e}\\

 \entry
    {2005 - 2009 \ \ \ \ \ }
    {B.S. Biology}
    {Truman State University}
    {Majoring in Biology\\
    \emph{Minor in Mathematical Biology}}\\
\end{entrylist}














\section{\faBook \ \  publications}

{\yearfont 2019}

\begin{itemize}

	\item {\mefont Dallas, TA},  M Saastamoinen, T Schulz, O Ovaskainen. 2019. The relative importance of local and regional processes to metapopulation dynamics. \textit{Journal of Animal Ecology}. doi: 10.1111/1365-2656.13141 

	\item \OA {\mefont Dallas, TA}, CJ Carlson, T Poisot. 2019. Testing predictability of disease outbreaks with a simple model of pathogen biogeography. \textit{Royal Society Open Science}. doi: 10.1098/rsos.190883

	\item {\mefont Dallas, TA}, Laine A-L, and Ovaskainen O. 2019. Detecting parasite associations within multi-species host and parasite communities. \textit{Proceedings of the Royal Society B} doi: 10.1098/rspb.2019.1109

	\item {\mefont Dallas, TA}, P{\"o}yry J, Leinonen R, Ovaskainen O. 2019. Temporal sampling and abundance measurement influences support for occupancy–abundance relationships. \textit{Journal of Biogeography} doi:10.1111/jbi.13718​

  \item Norberg,A, N Abrego Antia, F Guillaume Blanchet, FR Adler, BJ Anderson, J Anttila, MB Araújo, {\mefont TA Dallas}, D Dunson, J Elith, S Foster, R Fox, J Franklin, W Godsoe, A Guisan, B O'Hara, NA Hill, RD Holt, FKC Hui, M Husby, JA Kålås, A Lehikoinen, M Luoto, HK Mod, G Newell, I Renner, TV Roslin, J Soininen, W Thuiller, JP Vanhatalo, D Warton, M White, NE Zimmermann, D Gravel, and OT Ovaskainen. 2019. A comprehensive evaluation of predictive performance of 33 species distribution models at species and community levels. \textit{Ecological Monographs} doi:10.1002/ecm.1370
    
  \item Cornelius Ruhs, E, Borden, DM, {\mefont TA Dallas}, Pitman, E 2019. Do feather traits convey information about bird condition during fall migration? \textit{Wilson Journal of Ornithology} doi:10.1676/18-174

  \item {\mefont Dallas, TA}, AL Gehman, AA Aguirre, SA Budischak, JM Drake, MJ Farrell, R Ghai, S Huang, and I Morales-Castilla. 2019. Contrasting latitudinal gradients of body size in helminth parasites and their hosts. \textit{Global Ecology and Biogeography} doi: 10.1111/geb.12894

\item {\mefont Dallas, TA}, BA Han, CL Nunn, AW Park, PR Stephens, and JM Drake. 2018. Host traits associated with species roles in parasite sharing networks. \textit{Oikos} doi: 10.1111/oik.05602
 
\end{itemize}











{\yearfont 2018}
\begin{itemize}

\item {\mefont Dallas, TA}, BA Melbourne, and A Hastings. 2018. When can competition and dispersal lead to checkerboard distributions? \textit{Journal of Animal Ecology} doi: 10.1111/1365-2656.12913

\item {\mefont Dallas, TA} and A Hastings. 2018. Habitat suitability estimated by niche models is largely unrelated to species abundance. \textit{Global Ecology and Biogeography} doi: 10.1111/geb.12820

\item {\mefont Dallas, TA}, S Budischak, C Carlson, V Ezenwa, B Han, S Huang, AA Aguirre, and PR Stephens. 2018. Gauging support for macroecological patterns in helminth parasites. \textit{Global Ecology and Biogeography} doi: 10.1111/geb.12819


\item {\mefont Dallas, TA}, R Decker, AM Hastings. 2018. Multiple data sources and freely available code is critical when investigating species distributions and diversity: a response to Knouft (2018). \textit{Ecology Letters} doi: 10.1111/ele.13105

\item {\mefont Dallas, TA}, A Gehman, MJ Farrell. 2018. Variable bibliographic database access could limit reproducibility. \textit{BioScience} doi:10.1093/biosci/biy074

\item Park, AW, MJ Farrell, JP Schmidt, S Huang, {\mefont TA Dallas}, P Pappalardo, JM Drake, PR Stephens, R Poulin, CL Nunn, and TJ Davies. 2018. Characterizing the phylogenetic specialism-generalism spectrum of mammal parasites. \textit{Proceedings of the Royal Society B} doi: 10.1098/rspb.2017.2613

\item \OA {\mefont Dallas, TA}, JM Drake, and M Krkosek. Experimental evidence of a pathogen invasion threshold. \textit{Royal Society Open Science} doi: 10.1098/rsos.171975

\item {\mefont Dallas, TA} and T Poisot. 2018. Compositional turnover in host and parasite communities does not change network structure. \textit{Ecography} doi: 10.1111/ecog.03514

\end{itemize}




{\yearfont 2017}
\begin{itemize}

\item {\mefont Dallas, TA}, R Decker, AM Hastings. 2017. Species are not most abundant in the center of their geographic range or climatic niche. \textit{Ecology Letters} doi: 10.1111/ele.12860

\item Carlson, CJ, KR Burgio, {\mefont TA Dallas}, and WM Getz. The Mathematics of Extinction Across Scales: From Populations to the Biosphere. In \textit{Mathematics of Planet Earth: Quantitative Approaches to Issues of Current Interest}. (Eds: HG Kaper and FS Roberts) Springer. (\textit{forthcoming book})

\item \OA Carlson,CJ, KR Burgio, ER Dougherty, AJ Phillips, VM Bueno, CF Clements, G Castaldo, {\mefont TA Dallas}, CA Cizauska, GS Cumming, J Do\~na, NC Harris, R Jovani, S Mironov, O Muellerklein, HC Proctor, WM Getz. 2017. Parasite biodiversity faces extinction and redistribution in a changing climate. \textit{Science Advances} doi: 10.1126/sciadv.1602422

\item {\mefont Dallas, TA}, S Huang, C Nunn, AW Park, JM Drake. 2017. Estimating parasite host range. \textit{Proceedings of the Royal Society B}. 284:1861. doi:10.1098/rspb.2017.1250.

\item \OA {\mefont Dallas, TA}, AW Park, and JM Drake. 2017. Predicting cryptic links in host-parasite networks. \textit{PLoS Computational Biology}. 13(5): e1005557 doi:10.1371/journal.pcbi.1005557

\item \OA Evans, MV, {\mefont TA Dallas}, BA Han, CC Murdock, JM Drake. 2017. Data-driven identification of potential Zika virus vectors. \textit{eLife}. e22053. doi:10.7554/eLife.22053

\end{itemize}



{\yearfont 2016}
\begin{itemize}

\item \OA {\mefont Dallas, TA}, A Kramer, M Zokan, and JM Drake. 2016. Ordination obscures the influence of environment on plankton metacommunity structure. \textit{Limnology and Oceanography Letters}. 54-61. doi:10.1002/lol2.10028

\item {\mefont Dallas, TA}, AW Park, and JM Drake. 2016. Predictability of helminth parasite host range using information on geography, host traits and parasite community structure. \textit{Parasitology}. doi:10.1017/S0031182016001608

\item \OA {\mefont Dallas, TA} and JM Drake. 2016. Fluctuating temperatures alter environmental pathogen transmission in a \textit{Daphnia}-pathogen system. \textit{Ecology and Evolution} 00: 1-8. doi:10.1002/ece3.2539

\item \OA Stephens, P, Altizer, S, Smith, K, Aguirre, A, Brown, J, Budischak, S, Byers, J, {\mefont Dallas, TA}, Davies, J, Drake, J, Ezenwa, V, Farrell, M, Gittleman, J, Han, B, Huang, S, Hutchinson, R, Johnson, P, Nunn, C, Onstad, D, Park, A, Vazquez-Prokopec, G, Schmidt, J, and Poulin, R. 2016. The Macroecology of Infectious Diseases: A New Perspective on Global-scale Drivers of Pathogen Distributions and Impacts. \textit{Ecology Letters} 19(9): 1159-1171. doi: 10.1111/ele.12644

\item \OA {\mefont Dallas, TA} 2016. \textit{helminthR}: An R interface to the London Natural History Museum's Host-Parasite Database. \textit{Ecography} 39(4): 391-393. doi: 10.1111/ecog.02131 \href{https://github.com/ropensci/helminthR}{ \OD}

\item {\mefont Dallas, TA}, R Hall, and J Drake. 2016. Competition-mediated feedbacks in experimental multi-species epizootics. \textit{Ecology} 97(3):661-670. doi:10.1890/15-0305.1 \href{https://figshare.com/articles/R_code_to_reproduce_the_analyses_from_Dallas_Hall_and_Drake_2015_Competition-mediated_feedbacks_in_experimental_multi-species_epizootics_Ecology/3159349}{ \OD}

\item \OA {\mefont Dallas, TA}, M Holtackers, and J Drake. 2016. Costs of resistance and infection by a generalist pathogen. \textit{Ecology and Evolution} 6(6): 1737-1744. doi: 10.1002/ece3.1889 \href{}{ \OD}

\end{itemize}



{\yearfont 2015}

\begin{itemize}

\item \OA \ {\mefont Dallas, TA} and E Cornelius. 2015. Co-extinction in a host-parasite network: identifying key hosts for network stability. \textit{Nature Scientific Reports} doi: 10.1038/srep13185

\item Park, AW, C Cleveland, {\mefont TA Dallas}, and J Corn. 2015. Vector species richness increases hemorrhagic disease prevalence through functional diversity modulating the duration of seasonal transmission. \textit{Parasitology} 10: 1-6. doi: 10.1017/S0031182015000578

\item Presley SJ, {\mefont Dallas, TA}, Klingbeil, BT, Willig, MR. 2015. Phylogenetic signals in host-parasite associations for Neotropical bats and Nearctic desert rodents. \textit{Biological Journal of the Linnean Society} 116(2): 312-327. \href{http://datadryad.org/resource/doi:10.5061/dryad.bp62d}{ \OD}

\end{itemize}



{\yearfont 2014 and prior}

\begin{itemize}

\item \OA \ {\mefont Dallas, TA} and JM Drake 2014. Relative importance of environmental, geographic, and spatial variables on zooplankton metacommunities. \textit{Ecosphere} 5(9): art104 doi:10.1890/ES14-00071.1.

\item \OA \ {\mefont Dallas, TA} 2014. \textit{metacom}: an R package for the analysis of metacommunity structure. \textit{Ecography} 37(4):402-405. doi:10.1111/j.1600-0587.2013.00695.x

\item {\mefont Dallas, TA} and SJ Presley. 2014. Relative importance of host environment, transmission potential, and host phylogeny to the structure of parasite metacommunities. \textit{Oikos} 123: 866–874. doi:10.1111/oik.00707

\item \OA \ {\mefont Dallas, TA} and JM Drake 2014. Nitrate enrichment alters a Daphnia-microparasite interaction through multiple pathways. \textit{Ecology and Evolution} 4(3):243-250. doi: 10.1002/ece3.925

\item Kim, HJ, Cavanaugh, JE, {\mefont Dallas, TA}, and S For\'e. 2013. Model selection criteria for overdispersed data and their application to the characterization of a host-parasite relationship. \textit{Environmental and Ecological Statistics} doi:10.1007/s10651-013-0257-0

\item \OA \ {\mefont Dallas, TA} 2013. \textit{metacom}: Analysis of the 'Elements of Metacommunity Structure'. R package version 1.2. http://CRAN.R-project.org/package=metacom

\item {\mefont Dallas, TA} and S For\'e. 2013. Chemical attraction of \textit{Dermacentor variabilis} ticks parasitic to \textit{Peromyscus leucopus} based on host body mass and sex. \textit{Experimental and Applied Acarology} 61(2): 243-250. doi:10.1007/s10493-013-9690-x

\item {\mefont Dallas, TA}, S For\'e,  and  HJ Kim. 2012. Modeling the influence of \textit{Peromyscus leucopus} body mass, sex and habitat on immature \textit{Dermacentor variabilis} burdens. \textit{Journal of Vector Ecology}. 37(2):338-341.doi:10.1111/j.1948-7134.2012.00236.x

\item {\mefont Dallas, TA}, S For\'e and HJ Kim. 2010. Factors influencing immature \textit{Dermacentor variabilis} load on the white-footed mouse (\textit{Peromyscus leucopus}). \textit{Technical Report, Truman State University}.
\end{itemize}




\section{\faCode \ \  software}
\begin{entrylist}
 \entry
 {\href{http://cran.r-project.org/web/packages/metacom/}{\textbf{metacom}}}
 {Analysis of metacommunity structure} 
 {R package (author)}

 \entry
 {\href{http://github.com/ropensci/helminthR}{\textbf{helminthR}} \ \ }
 {Portal to London Natural History Museum host-helminth database}
 {R package (author)}

 \entry
 {\href{https://github.com/hmsc-r/HMSC}{\textbf{Hmsc}} \ \ }
 {Hierarchical modeling of species communities}
 {R package (author)}

 \entry
 {\href{http://github.com/cjcarlson/spatExtinct}{\textbf{spatExtinct}} \ \ }
 {Spatially interpolated extinction date estimation}
 {R package (contributor)}

\end{entrylist}





\section{\faVideoCamera \ \ \ selected presentations}

\begin{itemize}

\item {\mefont T Dallas}. \textit{Invited seminar at International University of Rijeka}. Hosted by Danijel Krismanic. June 8, 2019.

\item {\mefont T Dallas}. \textit{Invited seminar at Osnabr\"uck University}. Hosted by Frank Hilker. December 5, 2018.

\item {\mefont T Dallas}. \textit{Invited seminar at McGill University}. Hosted by Rowan Barrett. April 4, 2018. 

\item {\mefont T Dallas}. \textit{Invited seminar at University of Arkansas}. Hosted by John David Wilson. February 12, 2018. 

\item {\mefont T Dallas}. \textit{Invited seminar at Louisiana State University}. Hosted by Bret Elderd. January 30, 2018. 

\item {\mefont T Dallas}. \textit{Invited seminar at University of California - Los Angeles}. Hosted by Jamie Lloyd-Smith. January 9, 2018. 

\item {\mefont T Dallas}, B Melbourne, G Legault, A Hastings. Initial abundance and stochasticity influence species coexistence \textit{Society for Mathematical Biology}, July 19, 2017.

\item {\mefont T Dallas} and JM Drake. Using niche modeling to detect unobserved interactions in host-parasite networks. \textit{Ecological Society of America}, August 11, 2015.

\item JE Byers, P Pappalardo, JP Schmidt, PR Stephens, S Haas, C Nunn, JM Drake, and {\mefont T Dallas}. What parasite and host traits best explain the geographic range of mammal parasites and diseases? \textit{Ecological Society of America}, August 11, 2015.

\item {\mefont T Dallas} and JM Drake. Costs of resistance and infection in \textit{Daphnia} species exposed to a generalist microparasite. \textit{Ecology and Evolution of Infectious Disease Conference}. Fort Collins, CO. June 2014

\item  {\mefont T Dallas}, JM Drake, M Krkosek. Thresholds to pathogen invasion: theory + experiment. \textit{Ecological Society of America}. Sacramento, California. August 11, 2014

\item {\mefont T Dallas} and JM Drake. The Influence of Nitrate on Fungal Parasitism of \textit{Daphnia}. \textit{98th annual American Society for Microbiology (Southeastern Branch)}. October 2012.

\item {\mefont T Dallas}. Effects of competition and selective predation in a two-host system. \textit{Odum School of Ecology Graduate Student Symposium}. Athens GA. January 2011.

\item {\mefont T Dallas}. Thesis defense: An examination of variation in \textit{Dermacentor variabilis} burdens within and between host species. \textit{Truman State University}. August 2010.

\end{itemize}






\section{ \faInstitution \ \ \ teaching}

\begin{entrylist}

  \entry
    {2019}
		{Principles of Ecology (Biol 4253)}
    {Louisiana State University}

\end{entrylist}





%\section{\faDollar \ \ grants}
%\begin{entrylist}
% \entry
% {2019}
% {Money}
% {NSF}
%\end{entrylist}



\section{ \faUserPlus \ \ \ professional service}

\begin{entrylist}

 \entry
	 {2019-}
	 {Ecosphere}
	 {Subject Matter Editor}


\end{entrylist}

For information on my service, see my {\href{https://publons.com/author/904038/tad-dallas#profile}{Publons page}. I have served as a reviewer for the following journals:

\begin{multicols}{2}
\begin{itemize}
 \item African Journal of Wildlife Research
 \item American Naturalist
 \item Biological Conservation
 \item Ecography
 \item Ecology
 \item Ecology and Evolution
 \item Ecology Letters
 \item Ecological Complexity
 \item EcoHealth
 \item Ecosphere
 \item Functional Ecology
 \item Freshwater Biology
 \item Global Ecology and Biogeography
 \item Invertebrate Biology
 \item Journal of Animal Ecology
 \item Journal of Biogeography
 \item Journal of Vector Ecology
 \item Landscape Ecology
 \item Methods in Ecology and Evolution
 \item Nature Ecology \& Evolution
 \item Oecologia
 \item Oikos
 \item Philosophical Transactions B
 \item PLoS One
 \item Proceedings of the Royal Society B
 \item Scientometrics
 \item Theoretical Ecology
\end{itemize}
\end{multicols}



Further, I have served as webmaster for the following organizations:

\begin{itemize}
 \item  {\href{http://esa.org/disease}{Ecological Society of America - Disease Ecology section}}
 \item  {\href{http://diseasemacroecology.ecology.uga.edu/}{Macroecology of Infectious Disease - NSF Research Coordination Network}}
 \item  {\href{http://daphnia.ecology.uga.edu/ceesg}{Computational Ecology and Epidemiology Study Group - UGA}}
 \item  {\href{http://gsa.ecology.uga.edu}{Graduate Student Association - Odum School of Ecology}}
\end{itemize}






\section{\faUsers \ \ mentoring}
\begin{entrylist}

 \entry
 {2019 - }
 {Doctoral dissertation committee, LSU}
 {Jason Janeaux}


 \entry
 {2014}
 {Population Biology of Infectious Disease REU}
 {Trianna Humphries}

 \entry
 {2013}
 {Young Dawgs Program}
 {Mathieu Holtackers}

\end{entrylist}















\iffalse
\section{\faGlobe \ \ professional affiliations}
\begin{entrylist}

	\entry
	{2017 - \ \ \ \ \ \ \ \ \ \ }
  {Society for Mathematical Biology}
  {}

	\entry
  {2016 - \ \ \ \ \ \ \ \ \ \ }
  {Association for the Sciences of Limnology and Oceanography}
  {}

	\entry
  {2012 - }
  {Ecological Society of America member}
  {Aquatic Ecology and Disease Ecology sections}

	\entry
  {2010 - }
  {Phi Kappa Phi member}
  {Academic honor fraternity}

\end{entrylist}
\fi


\end{document}
